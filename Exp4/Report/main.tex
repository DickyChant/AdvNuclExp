\documentclass{article}
\usepackage{ctex}
\usepackage{geometry}
\geometry{top = 2cm, left = 1cm, right = 1cm, bottom = 2cm}
\usepackage{amsmath,amssymb,amsthm,amsfonts}
\usepackage{abstract}
\usepackage{siunitx}
\usepackage{graphicx}
\usepackage{booktabs}
\usepackage{appendix}
\usepackage{hyperref}
\usepackage{lscape}

\renewcommand{\appendixpagename}{附录}


\title{利用逆矩阵法解$\gamma$谱}
\author{钱思天 2001112187}
\begin{document}
    \maketitle
    \begin{abstract}
        本实验通过逆矩阵法,根据对已知$^{60}\text{Co},^{137}\text{Cs}$源的测量,进行了
        \newline
        \newline
        {\emph{ 关键词:\ 反符合、$\gamma$射线谱、康普顿效应 }\rm}

    \end{abstract}

    \section{实验介绍}
    \subsection{实验原理}
  

\section{致谢}
    非常感谢赵捷老师的实验指导,也非常感谢和童星昱同学的讨论。
    \clearpage
    \appendix
    \appendixpage
    \section{思考题}
    \begin{enumerate}
        \item 提高伽马能谱的峰康比后,可以有效提高存在高能强峰时探测低能弱峰的能力。即如果低能弱峰落在高能强峰的康普顿坪上时,可以避免其被康普顿坪所
        淹没,使能谱更加干净。
        \item 通过设置阈值,卡掉全能峰。
        \item 偶然符合的计数率为
        \begin{equation}
            (N_1+N_2)-N_1N_2\tau
        \end{equation}
    \end{enumerate}
\end{document}