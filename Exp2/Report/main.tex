\documentclass{article}
\usepackage{ctex}
\usepackage{geometry}
\geometry{top = 2cm, left = 1cm, right = 1cm, bottom = 2cm}
\usepackage{amsmath,amssymb,amsthm,amsfonts}
\usepackage{abstract}
\usepackage{siunitx}
\usepackage{graphicx}
\usepackage{booktabs}
\usepackage{appendix}
\usepackage{hyperref}

\renewcommand{\appendixpagename}{附录}


\title{高纯锗(HPGe)$\gamma$射线谱仪}
\author{钱思天 2001112187}
\begin{document}
    \maketitle
    \begin{abstract}
        本实验利用了高纯锗(HPGe)谱仪探测器,对$^{60}\text{Co},^{152}\text{Eu}$源、环境本底以及矿渣样本进行测量。
        通过对$^{152}\text{Eu}$源的测量,对探测系统进行了相对效率刻度,而后,通过对$^{60}\text{Co}$源的测量,对实验测量系统做了绝对效率刻度。而后,通过调节放大倍数并重新刻度,让探测器能够探测矿渣中的$\gamma$射线,进而判断出矿渣中所含有的元素。
        \newline
        \newline
        {\emph{ 关键词:\ 高纯锗$\gamma$射线谱仪、探测效率刻度、未知成分探测 }\rm}

    \end{abstract}

    \section{背景简介}
    \subsection{高纯锗$\gamma$射线谱仪}
    $\gamma$射线是原子核衰变或裂变时放出的辐射,本质上它是一种能量比可见光和$X$射线高得
多的电磁辐射。利用$\gamma$射线与物质相互作用的规律,人们已设计出多种类型的$\gamma$射线的探测
器。用于测量$\gamma$射线能量和强度的仪器称为$\gamma$能谱仪,谱仪一般有射线探测器和电子学系统
两大部分。最常用的有 NaI(Tl)闪烁谱仪和高纯锗(HPGe)半导体谱仪。闪烁探测器是利用某些
物质在射线作用下发光的特性来探测射线的仪器。HPGe 探测器是利用$\gamma$光子与探测介质原
子发生相互作用产生次级电子,通过次级电子在介质中的电离效应来探测射线的仪器。本实
验介绍一种高分辨率的高纯锗$\gamma$谱仪。
高纯锗探测器(High-Purity Germanium 简称 HPGe 探测器)实质上是利用纯度极高的锗制
成的 P-N 探测器。由于锗的纯度极高,也就是杂质浓度很低,因而电阻率很大,可得到较厚
的耗尽层,又由于锗的原子序数高,适合于探测$\gamma$射线。HPGe 探测器具有能量分辨率极高
的优点,所以它被广泛应用于$\gamma$射线能谱的测量,使$\gamma$谱学的面貌发生了根本的变化。
近年来,在核谱学、核反应、核工程和核技术应用等方面,HPGe 探测器已成为分析复
杂$\gamma$能谱的不可缺少的工具。
\subsection{实验目的}
\begin{enumerate}
    \item 了解高纯锗(HPGe)γ射线谱仪的原理、一般操作以及数据采集、处理的方法等;
    \item 利用$^{60}\text{Co}$,$^{152}\text{Eu}$做探测效率刻度;
    \item 测定未知样品的种类、活度。
\end{enumerate}
\section{}
\subsection{实验装置}
本实验用到的实验装置比较简单,有:
\begin{enumerate}
    \item 高纯锗谱仪一套;
    \item NIM机箱、插件式高压电源、低压电源、主放大器各一个;
    \item 多道数据采集以及微机系统一套。
    \item $^{60}\text{Co},^{152}\text{Eu}$放射源各一个;
    \item 未知矿渣样品及固定装置一套。
\end{enumerate}
\subsection{实验操作}
鉴于上一位同学实验结束时的状况,实验步骤调整如下:
\begin{enumerate}
    \item 连接电子学线路,
检查无误后给高纯锗探头加高压至$3500\si{V}$。并且沿用上一位同学成功的放大倍数$(50\times1.12)$。
    \item 测定环境本底与矿渣样品能谱,各20分钟。
    \item 测量$^{152}\text{Eu}$十分钟,利用此数据及接下来测量的$^{60}\text{Co}$数据进行刻度。
    \item 将主放大器的放大倍数调至两倍$(100\times 1.12)$,而后分别测量$^{60}\text{Co},^{152}\text{Eu}$及环境本底各10分钟,并利用他们进行刻度。
\end{enumerate}

\end{document}