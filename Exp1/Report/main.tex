\documentclass{article}
\usepackage{ctex}
\usepackage{geometry}
\geometry{top = 2cm, left = 1cm, right = 1cm, bottom = 2cm}
\usepackage{amsmath,amssymb,amsthm,amsfonts}
\usepackage{abstract}

\title{用正比计数器测量X射线的吸收和特征谱}
\author{钱思天 2001112187}
\begin{document}
    \maketitle
    \begin{abstract}
        本实验利用了正比探测器和多道计数器,对经由$^{238}\text{Pu}$放射源激发的不同样品特征X射线进行了测量。通过对铁、钴、锗、铜、锌等元素的特征X射线测量,
        结合参考$K_\alpha$能谱对多道计数器进行了能量刻度。并利用刻度后的测量系统测量了三个未知样品的特征X射线能量,继而确定了元素种类。而后,通过添加不同厚度的
        铝吸收片的铜特征X射线的测量结果,测得了铝对X射线的线性吸收系数。最后,利用了$^{55}\text{Fe}$的特征X射线的测量结果,评估了正比探测器的分辨本领。
        \newline
        \newline
        {\emph{ 关键词:\ 正比计数器、特征X射线、线性吸收系数 }\rm}

    \end{abstract}

    \section{实验简介}
    由低能$\gamma$源等照射样品使样品激发产生的源激发X射线,作为样品的本征X射线,可以被用来辨认样品的元素种类。
        但是其产额较低,因此需要有高分辨的探测器。正比探测器是
    \section{数据处理}
\end{document}